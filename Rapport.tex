\documentclass[a4paper]{report}	

% Chargement d'extensions
\usepackage[utf8]{inputenc}     % Pour utiliser les lettres accentuées
\usepackage[T1]{fontenc}
\usepackage{lmodern}
\usepackage{amsmath}
\usepackage{amssymb}
\usepackage{cases}
\usepackage{mathrsfs}
\usepackage [francais]{babel}     % Pour la langue française
\usepackage{graphicx}	%pour l'insertion de figures 
\usepackage{verbatim}		%pour le texte brut 
\usepackage{moreverb}		%text brut avec tab
\usepackage{listings} %aussi pour texte brut
\usepackage{color}
\usepackage{subfigure} %pour des dous-figures
\usepackage{multicol}
\usepackage{geometry}
\geometry{margin=1in} % for example, change the margins to 1 inches all round

\addto\captionsfrench{\renewcommand{\chaptername}{Partie}}

% Informations le titre, le(s) auteur (s), la date
\title {Rapport - Data mining}
\author {Brendan LE LAIN}
\date{\today}

% Début du document
\begin {document}
 
\pagestyle{headings}
 
\maketitle
 
\chapter{Rappel sur les méthodes utilisées}

\section{L'analyse factorielle}
Les techniques factorielles permettent à la fois de représenter en deux ou trois dimensions, le plus fidèlement possible, les individus d'une population, mais aussi de détecter les liaisons entre les variables ainsi que les variables séparant le mieux les individus. Elles font appel à l'algèbre linéaire. C'est un outil très visuel : un simple coup d'oeil permet de localiser les grandes masses d'individus, de détecter les individus exceptionnels et de repérer d'éventuels groupes isolés d'individus. Les techniques factorielles sont aussi un puissant outil de réduction des dimensions d'un problème, qui permet de diminuer le nombre de variables étudiées en perdant le moins possible d'information.

\subsection{Analyse en Composantes Principales}

\subsubsection{Objectifs}
L'objectif d'une Analyse en Composantes Principales (ACP) est de résumer un tableau de données individus $\times$ variables, lorsque les données sont toutes numériques. L'ACP permet d'étudier les ressemblances entre individus du point de vue de l'ensemble des variables et dégage des profils d'individus. Elle permet également de réaliser un bilan des liaisons linéaires entre variables à partir des coefficients de corrélations. Ces études étant réalisées dans un même cadre, elles peuvent être reliées, ce qui permet de caractériser les individus ou groupes d'individus par les variables et d'illustrer les liaisons entre variables à partir d'individus caractéristiques.

Ainsi, l'ACP permet, à partir de $p$ variables numériques analysées, de construire $m$ ($\leq p$) autres variables, appelées composantes principales ou facteurs, qui sont des combinaisons linéaires des variables analysées, et qui présentent d'intéressantes caractéristiques : 
\begin{itemize}
\item les composantes principales sont ordonnées selon l'information qu'elles restituent, la première étant celle qui restitue le plus d'information ;
\item on sait quelle part d'information restitue chaque composante principale, et des critères permettent de décider combien de composantes principales il est judicieux de conserver ;
\item les composantes principales sont des vecteurs indépendants, c'est-à-dire des variables non corrélées linéairement entre elles (l'ACP n'est donc pas affectée par la présence de données corrélées).
\end{itemize}

\subsubsection{Atouts de l'ACP}
La représentation des variables dans un plan factoriel est le moyen le plus intuitif et le plus pratique de repérer les variables liées entre elles ou au contraire opposées.
Un autre intérêt de l'ACP est de permettre, par sa représentation graphique, de contrôler le résultat d'une classification obtenue, soit indépendamment, soit à partir des composantes principales. Cette représentation permet de :
\begin{itemize}
\item vérifier visuellement la pertinence de la classification ;
\item éventuellement aider à choisir le nombre de classes le plus approprié ;
\item isoler facilement certains individus dont les données paraissent aberrantes et proviennent peut-être d'une erreur de mesure ou de saisie ;
\item sélectionner les individus les plus typiques d'une classe, ou au contraire ceux qui sont proches d'une classe voisine et vers laquelle ils peuvent évoluer.
\end{itemize}

\subsubsection{Écueils à éviter}
\begin{itemize}
\item Il ne faut pas superposer l'espace des individus et celui des variables ;
\item dans un plan factoriel, la proximité de deux variables n'a de sens que si elles sont proches du cercle des corrélations ;
\item le premier plan factoriel n'est pas le seul à dévoiler des informations intéressantes, on peut aussi avec profit croiser le premier et troisième, etc. ;
\item il faut éviter qu'un individu (ou un petit groupe d'individus) ait une trop forte contribution aux premiers axes, le pire étant qu'un axe soit complètement expliqué par un seul individu.
\end{itemize}


\subsection{Analyse Factorielle des correspondances}

\subsubsection{Objectifs}
L'analyse factorielles des correspondances (AFC) permet de résumer et de visualiser un tableau de contingence, c'est-à-dire un tableau croisant deux variables qualitatives. Les objectifs de l'AFC sont les suivants :
 \begin{itemize}
 \item comparer les profils lignes entre eux
 \item comparer les profils colonnes entre eux
 \item interpréter les proximité entre les lignes et les colonnes, c'est-à-dire visualiser les associations des modalités des deux variables
 \end{itemize}

\subsection{Analyse des Correspondances Multiples}

\subsubsection{Objectifs}
L'analyse des correspondances multiples (ACM) est l'extension de l'AFC au cas de plus de deux variables. L'ACM permet d'étudier les ressemblances entre individus du point de vue de l'ensemble des variables et de dégager des profils d'individus. Elle permet également de faire un bilan des liaisons entre variables et d'étudier les associations de modalités.

\subsubsection{Atouts de l'AFC et de l'ACM}
\begin{itemize}
 \item les facteurs sont les variables numériques qui séparent le mieux les modalités des variables qualitatives étudiées ;
 \item elles permettent de vérifier visuellement que des modalités voisines par définition sont proches dans le plan factoriel ;
 \item l'AFC et l'ACM permettent de transformer des variables qualitatives (initiales) en variables quantitatives (les projections des modalités des variables initiales sur les axes factoriels), ce qui peut être un pré-traitement avant une nouvelle étude (analyse discriminante par exemple) ;
 \item elles permettent de traiter simultanément les variables quantitatives (en les discrétisant) et qualitatives ; 
 \item elles permettent de représenter simultanément individus et modalités sur un même plan ;
 \end{itemize}
 
 \subsubsection{Aspects à vérifier}
 En AFC et ACM, il est très important de vérifier qu'il n'y a pas de modolités rares (avec un faible effectif), car ces méthodes apportent beaucoup d'importance à ces modalités. 
 
 
 \chapter{Problème 1}
 
\section{Description des données et objectif de l'étude}

\emph{Mammographic Mass data set} présente 3 mesures issues de mammographies (variables \verb|shape| , \verb|density| et \verb|margin|) réalisées sur 961 patientes ayant une tumeur. L'âge (variable \verb|age|) ainsi qu'un jugement de spécialistes sur la gravité de la tumeur (variable \verb|assessment|) sont aussi disponibles. Les données sont complétées par la sévérité effective de la tumeur : bénigne ou non (variable \verb|severity|).

L'objectif de l'étude est de voir si les 3 variables \verb|shape| , \verb|density| et \verb|margin| sont liées entre elles (il s'agira par exemple de répondre à des questions comme ''les tumeurs de forme ronde ont-elles en général une frange particulière?'' ou encore ''Peut-on identifier des groupes d'individus possédant en général certaines modalités?''. L'objectif est dans un deuxième temps d'identifier des profils d'individus possédant une forte probabilité que la tumeur soit maligne. 

Le code \verb|R| et \verb|SAS| que nous avons utilisé est fournit en annexe A.

\subsection{Premier aperçu des données}
Le fichier des données brutes ne présente pas le nom des variables, utilise la virgule comme caractère de séparation et le point d'interrogation pour coder les valeurs manquantes. 

Les données comprennent 131 lignes ayant au moins une valeur manquante. On décide de supprimer toutes ces lignes.

Pour la variable \verb|assessment| (qui va de 1 à 5) on observe quelques données abérantes prenant les valeurs 0 ou 6. On supprime ces lignes. On a aussi une ligne avec la valeur 55, on peut supposer que c'est une erreur de frappe, on la remplace par la valeur 5.

\subsubsection{Age}

La variable \verb|age|, contrairement aux autres qui sont qualitatives, est continue. On peut penser que cette variable admet une liaison significative avec la variable cible \verb|severity|, puisqu'on a tendance à penser que le risque d'une tumeur maligne augmente avec l'âge. On décide donc de continuer en la discrétisant.

Dans notre jeu de données, cette variable a une allure normale, avec une moyenne à environ 56 ans. Pour mieux voir l'influence de l'âge, on le découpe en 11 quantiles. (Le nombre 11 est complètement arbitraire, c'est simplement celui qui nous a paru, après plusieurs essais avec d'autres valeurs, le plus intéressant). Puis nous croisons ces quantiles avec la variable \verb|severity|. 

\begin{figure}[!ht]
	\centering
     	\includegraphics[scale=0.4]{PB1/Plot/age1.png} 
       	\includegraphics[scale=0.5]{PB1/Plot/age2.png} 
       	\caption{À gauche, le découpage en 11 classes avec les quantiles permet d'avoir des effectifs homogènes pour chaque modalité. À droite, le croisement avec la sévérité de la tumeur montre un lien net.} 
\end{figure}


On retrouve bien ce que notre intuition nous soufflait puisque la proportion de tumeurs malignes augmentent nettement avec l'âge. Plus précisément, on peut identifier 4 ensembles :
\begin{itemize}
\item un premier ensemble d'individus ayant moins de 35 ans et pour lesquels le risque de tumeur maligne est très faible (environ 7.5 \% des cas)
\item un ensemble formé d'individus agés entre 36 et 51 ans où la proportion de tumeur maligne est autour de 30\%)
\item on observe ensuite un groupe d'individus agés de 52 à 65 ans où le risque est important (50\% des tumeurs sont malignes)
\item enfin les individus de plus de 66 ans présentent une tumeur maligne dans environ 80\% des cas.
\end{itemize} 

La variable \verb|age| nous apparaît donc importante. Nous choisissons de la prendre en compte dans l'analyse à suivre en la rendant qualitative suivant le découpage qui vient d'être identifié. Elle pourra donc prendre les 4 modalités suivantes : \verb|<=35|, \verb|36-51|, \verb|52-65| et \verb|>=66|.


\subsubsection{Pouvoir discriminant des variables}
Avant d'étudier les autres variables, il est intéressant d'étudier le pouvoir discriminant de l'ensemble des variables, mesuré par la valeur absolue du V de Cramer de la variable explicative avec la variable à expliquer. 

\begin{figure}[!ht]
	\centering
     	\includegraphics[scale=0.2]{PB1/Plot/cramer.png} 
\end{figure}

On observe que \verb|density| a une valeur beaucoup plus faible que les autres variables. On peut déjà voir ici que cette variable ne donnera pas beaucoup d'information sur la sévérité de la tumeur. À l'opposé, \verb|assessment| a une valeur forte du V de Cramer, ce qui montre que le jugement apporté par les spécialistes sur la sévérité de la tumeur est efficace. Les trois autres variables viennent ensuite, avec une valeur qui reste haute, montrant par là leur pouvoir prédictif de la sévérité de la tumeur. 

Regardons à présent de plus près chaque variable.


\subsubsection{Assessment}
Cette variable est ordianle, constituée d'un jugement de spécialistes ayant attribué un degré de la sévérité de la tumeur : de 1 (tumeur bénigne) jusqu'à 5 (tumeur très probablement maligne). On s'attend donc à ce qu'une valeur élevée (4 ou 5) soit attribuée à une tumeur qui se révèlerait être maligne, et une valeur faible à une tumeur bénigne.

\begin{figure}[!ht]
	\centering
     	\includegraphics[scale=0.2]{PB1/Plot/assess.png} 
\end{figure}


On observe que les individus sont très majoritairement classés dans les modalités 4 et 5 (dans plus de 96\% des cas). Cependant, la sévérité de la tumeur est alors en quelque sorte surestimée, puisqu'il arrive souvent (presque 78\% des cas) qu'une tumeur classée 4 se révèle bénigne. Ceci paraît valider l'hypothèse faite dans le commentaire des données brutes, à savoir qu'une tumeur réellement maligne et pour laquelle une biopsie (opération consistant à enlever une partie de l'organe) est nécessaire est difficile à prédire. Les spécialistes ont alors tendance à être prudents et à réaliser plus de biopsies qu'il serait en fait nécessaire.   

Cette première analyse nous confirme ainsi dans l'idée qu'une étude statistique des résultats de la mammographie pourrait aider à mieux identifier les profils de tumeurs à risque.


\subsubsection{Shape}
Vient ensuite la variable \verb|shape| qui est qualitative et prend les modalités \verb|round|, \verb|oval|, \verb|lobular| et \verb|irregular|. 

\begin{figure}[!ht]
	\centering
     	\includegraphics[scale=0.2]{PB1/Plot/shape.png} 
\end{figure}

À la vue du tableau obtenu en croisant cette variable avec \verb|severity|, on est conforté dans l'idée que \verb|shape| est un facteur important de la sévérité de la tumeur et que les 4 modalités sont classées par ordre de risque. On observe le phénomène suivant :
\begin{itemize}
\item les modalités \verb|round| et \verb|oval| sont à risque faible (environ 17\% de tumeurs malignes)
\item \verb|lobular| présente un risque moyen (50\% de tumeurs malignes)
\item \verb|irregular| présente un risque élevé (plus de 78\% des tumeurs possédant cette modalité se révèlent être des tumeurs malignes)
\end{itemize}


\subsubsection{Margin}
La variable \verb|margin| prend les 5 modalités \verb|circumscribed|, \verb|microlobulated|, \verb|obscured|, \verb|ill-defined| et \verb|spiculated|. 

\begin{figure}[!ht]
	\centering
     	\includegraphics[scale=0.2]{PB1/Plot/margin.png} 
\end{figure}

Comme précedemment, on croise cette variable avec \verb|severity| pour observer qu'elles sont liées significativement. En particulier : 
\begin{itemize}
\item la modalité \verb|circumscribed| présente une proportion faible de tumeurs malignes (12\%)
\item un saut très prononcé apparait pour les 4 autres modalités (de 60 à 83\% de cas de tumeurs malignes)
\item dans ce groupe à risque élevé, on peut séparer les modalités  \verb|microlobulated|, \verb|obscured| et \verb|ill-defined| qui présentent des tumeurs malignes dans environ 65\% des cas. La dernière modalité - \verb|spiculated| - est plus risquée : 83\% de tumeurs malignes
\end{itemize}
 

\subsubsection{Density}
On trouve enfin la variable \verb|density| qui prend 4 modalités : \verb|high|, \verb|iso|, \verb|low| et \verb|fat-containing|. 

\begin{figure}[!ht]
	\centering
     	\includegraphics[scale=0.2]{PB1/Plot/density.png} 
\end{figure}

Comme repéré par la valeur du V de Cramer, cette variable ne semble pas très informative concernant la sévérité de la tumeur. En effet, la grande majorité des individus (près de 91\%) sont classés dans la modalité \verb|low|. De plus, le croisement avec \verb|severity| montre que pour chaque modalité environ 50\% des tumeurs se révèlent etre malignes. Cette variable n'ajoute pas ou peu d'information sur la sévérité. Elle pourrait meme nous gener dans la réalisation d'une analyse des correspondances du fait du manque d'effectifs de certaines des modalités. On préfère donc se passer de cette variable pour l'analyse en concluant déjà qu'elle ne permet d'établir un lien clair avec la sévérité de la tumeur.

  \subsection{Analyse des correspondances multiples (ACM)}
  Étant donné le caractère des données, l'objectif et la première analyse ci-dessus, on réalise une ACM sur les variables \verb|age|, \verb|shape| et \verb|margin|. 
  
  \begin{figure}[!ht]
	\centering
     	\includegraphics[scale=0.5]{PB1/Plot/Rplot2.eps}

	\caption{représentation des modalités après l'ACM. Les modalités actives sont en rouge, les modalités supplémentaires en vert.}
\end{figure}
   
La figure 2.2 permet de tirer un certain nombre d'observations. La première est que l'on peut qualifier l'axe 1 d'axe de sévérité, les modalités supplémentaires \verb|benign| et \verb|malignant| étant bien représentées sur cet axe, symétriquement à 0. On note ensuite que l'âge est bien représenté par l'axe 1 : plus l'individu est âgé plus la tumeur est susceptible d'être maligne. La représentation de la variable \verb|shape| recoupe notre analyse précédente : les modalités \verb|round| et \verb|oval| préfigure en général une tumeur bénigne ; la modalité \verb|lobular| présente un risque moyen-élevé et la forme \verb|irregular| est par contre signe d'une tumeur maligne. On retrouve aussi ce que l'on avait dit concernant la variable \verb|margin| : une modalité sans risque important (\verb|circumscibed|) et 4 modalités risquées, dans cet ordre, \verb|microlobulated|, \verb|obscured|, \verb|ill-defined| et \verb|spiculated|.

D'autre part, on retrouve également la surestimation du risque par la variable \verb|assessment| : le niveau 4 est proche du caractère bénin de la tumeur.

On peut ensuite repérer un profil type de tumeur bénigne : ce sera un individu âgé de moins de 51 ans, présentant les modalités \verb|round| ou \verb|oval| pour la forme de la tumeur et la modalité \verb|circumscribed| pour la frange.
De même un profil à risque très élevé de tumeur maligne est un individu de plus de 66 ans, avec une tumeur de forme irrégulière et une modalité \verb|ill-defined| pour la frange.
 
 Pour aller un plus loin dans l'analyse, on représente à la figure 2.3 le rapport de corrélation entre les dimensions et les variables, ceci afin de mieux connaître les variables les plus liées aux axes.
 \begin{figure}[!ht]
	\centering
     	\includegraphics[scale=0.5]{PB1/Plot/Rplot3.eps}

	\caption{Représentation des variables actives (en rouge) et illustratives (en vert).}
\end{figure}
On voit que \verb|margin| contribue le plus à la création des axe 1 et 2, suivi de près par \verb|shape|. On peut ainsi penser qu'un lien fort uni ces 2 variables. D'où l'idée de raliser une analyse des correspondances simples pour mieux comprendre ce lien.

  \subsection{Analyse factorielle des correspondances (AFC)}
   
   \subsubsection{Shape et margin}
   L'AFC sur les variables \verb|shape| et \verb|margin| permet de visualiser le lien fort qui uni effectivement ces 2 variables : le test du Chi 2 d'indépendance est rejeté très largement, la valeur du V de Cramer est 0.4592. 

      \begin{figure}[!ht]
	\centering
     	\includegraphics[scale=0.5]{PB1/Plot/Rplot4.eps}

	\caption{Représentation des modalités après AFC}
\end{figure}

   On identifie clairement 3 profils : 
   \begin{itemize}
   \item \verb|round|, \verb|oval| et \verb|circumscribed|
   \item \verb|microlobulated| et \verb|lobular|
   \item \verb|obscured|, \verb|ill-defined|, \verb|spiculated| et \verb|irregular|
   \end{itemize}
   
   
      \subsubsection{Shape et age}
   On réalise aussi une AFC sur les variables \verb|shape| et \verb|age|. Ce qui nous permet de visualiser la liaison entre ces 2 variables : le test du Chi 2 d'indépendance est rejeté très largement, même si la valeur du Chi 2 n'est pas aussi élevée que dans le cas précédent ; le V de Cramer est de 0.2178. 

      \begin{figure}[!ht]
	\centering
     	\includegraphics[scale=0.5]{PB1/Plot/Rplot5.eps}

	\caption{Représentation des modalités après AFC}
\end{figure}

   On identifie 3 profils : 
   \begin{itemize}
      \item \verb|35-51|, \verb|oval| et \verb|round|
   \item \verb|51-66| et \verb|lobular|
      \item \verb|>=66| et \verb|irregular|
   \end{itemize}
   La modalité \verb|<=35| est seule nettement à droite sur le premier axe. 

     \subsubsection{Margin et age}
   Enfin, l'AFC sur les variables \verb|margin| et \verb|age| permet de visualiser le lien qui uni effectivement ces 2 variables : le test du Chi 2 d'indépendance est rejeté très largement, la valeur du Chi 2 est du même ordre que pour le cas de \verb|shape| et \verb|age| ; la valeur du V de Cramer est de 0.2559. 

      \begin{figure}[!ht]
	\centering
     	\includegraphics[scale=0.5]{PB1/Plot/Rplot6.eps}

	\caption{Représentation des modalités après AFC}
\end{figure}

   On identifie 2 profils : 
   \begin{itemize}
      \item \verb|<35| \verb|35-51| et \verb|circumscribed|
   \item \verb|51-66|, \verb|>66|, \verb|microlobulated|, \verb|obscured|, \verb|ill-defined|, et \verb|spiculated|
   \end{itemize}
   
   
  \subsection{Conclusion}
  Par une ACM puis différentes ACP nous sommes parvenus à identifier les liens existants entre les modalités des 3 variables \verb|shape|, \verb|age| et \verb|margin| et à conforter l'idée que ces variables permettaient effectivement d'identifier une tumeur maligne. On a déjà noté que les spécialistes avaient tendance à surestimer la sévérité de la tumeur. Par exemple, parmi les individus classés 4 par la variable \verb|assessment|, 66 étaient agés de moins de 35 ans. Sur ces 66 tumeurs, 64 se sont révélées être bénignes. Beaucoup possédaient en effet des modalités \verb|shape| et \verb|margin| que nous avons identifié comme étant peu risquées.   
  
   L'étape suivante serait donc, à partir d'un profil donné, d'essayer d'établir un risque que la tumeur soit maligne. Cela pourrait permettre de compléter le jugement des spécialistes. Pour réaliser un tel travail de prévision, on pense à la technique du scoring réalisée avec une analyse discriminante linéaire ou une régression logistique\footnote{Pour la mise en pratique de cette méthode, on pourra consulter le chapitre 17 de l'ouvrage de Stéphane TUFFÉRY}.    
  
  
   \chapter{Problème 4}
 
\section{Description des données et objectif de l'étude}
Le jeu de données \emph{viruses} présentent 61 virus présentant des paarticules en forme de tige et susceptibles d'infecter diverses plantations (tabac, tomates, comcombres par exemple). À chaque virus correspond 18 mesures qui sont le nombre d'acides aminés par molécules de protéine d'enveloppe\footnote{Pour des détails sur les données, on pourra consulter l'article de Fauquet disponible à cette adresse : $http://horizon.documentation.ird.fr/exl-doc/pleins_textes/pleins_textes_7/b_fdi_55-56/010022153.pdf$}. 

Les données sont rangées dans l'ordre de quatre types de virus : Hordeviruses (3), Tobraviruses (6), Tobamoviruses (39) et 'furoviruses' (13). 

L'objectif de l'étude est de repérer des ressemblances entre différents virus et d'essayer d'interpréter ces ressemblances en fonction des varaibles disponibles. 

\subsection{Premier aperçu des données}
Vu la nature des données, une première approche est d'ajouter une variable qui sera la somme des valeurs des autres variables. En effet, une variable représentant le nombre d'acides aminés pour 18 protéines, faire le total de ces acides aminés pourraient apporter une première idée de regroupement des virus.

On présente ci-dessous l'histogramme obtenu pour cette nouvelle variable ainsi que son boxplot en fonction des 4 types de virus.

 \begin{figure}[!ht]
	\centering
     	\subfigure[Histogramme du nombre total d'acides aminés]{ \includegraphics[scale=0.38]{PB4/RplotHist.eps} }
	\subfigure[Boxplot par type de virus]{ \includegraphics[scale=0.38]{PB4/RplotBoxplot.eps} }

\end{figure}

Sur l'histogramme un groupe autour de la valeur 158 se détache clairement. Un autre groupe semble se former à l'extrême droite. Entre les valeurs 175 et 200 il est assez difficile de voir si un ou plusieurs groupes apparaissent.  

Le boxplot est intéressant dans le sens on peut observer les virus exceptionnels. Pour le type \emph{furoviruses}, ce sont les virus 56 (total de 170) et le virus 55 (198). Pour le type \emph{tobamoviruses} on remarque surtout le virus 11 avec une valeur particulièrement faible (122).

Cette première petite étude nous permet de retenir ces 3 virus, pour éventuellement les laisser de côté à un moment. Elle montre aussi qu'excepté ces 3 virus, la division en 4 groupes de virus semblent bien s'opérer. Il est à présent intéressant de voir ce qui différencie ces différents groupes, d'en savoir un peu plus sur les variables. Pour cela, on réalise une ACP. 

\section{Analyse en composantes principales}






 \begin{appendix}
 \chapter{Problème 1}
  \label{Pb1}
 On présente ici le code utilisé pour l'étude du premier problème. Nous avons utilisé conjointement \verb|R| et \verb|SAS|. Si il y a certaines répétitions, cela permet de voir les différences de syntaxe et de sorties.
 
 Pour \verb|R| : 
\lstset{frame=single, xrightmargin =1cm , xleftmargin =1 cm}
\lstinputlisting[firstline=2, lastline=193]{./PB1/pb1_OK.R}

Pour \verb|SAS| : 
\lstinputlisting[]{./PB1/pb1.sas}
\end{appendix}



\end{document}